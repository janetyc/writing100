\documentclass[a4paper]{article}
% Preamble Area

\usepackage{geometry} 
\usepackage{graphicx}


 \title{\textbf{Writing 100 days}}
\author{Janet Yi-Ching Huang}


\begin{document}
\maketitle

\section{Day 2: Writing Structure Extraction}
The quality of paper structure affects reading comprehension. To help a writer improve she/his writing, we need to assess the structure of a paper. However, evaluating the structure of a document is challenging for a novice reader. It allows a reader to understand, analyze, and judge ideas delivered in a paper. To aid structure assessment, we provide an approach which identifies the main ideas and extracts a causal relation between two ideas.

\section{Day 1: Crowdsourced Writing}
Recent studies have decomposed writing process into a series of micro-tasks in crowdsourcing domain. Soylent \cite{Bernstein:UIST10} splits writing projects into multiple stages and allows crowd workers to shorten, proofread, and edit a document. MicroWriter [cited] enable collaborative writing by breaking the task of writing into three types of micro-tasks--generating ideas, labeling ideas to organize them, and writing paragraphs given a few related ideas.

\bibliographystyle{abbrv} 
\bibliography{writing100}

\end{document} 