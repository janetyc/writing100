\documentclass[a4paper]{article}
% Preamble Area

\usepackage{geometry} 
\usepackage{graphicx}


 \title{\textbf{Writing 100 days}}
\author{Janet Yi-Ching Huang}


\begin{document}
\maketitle
\section{Day 7: Facilitating Research writing}
Writing is an important academic activity for researchers. However, it imposes significant challenges for writers, especially for novice student writers. The challenge is that they not only present their ideas in disciplinary discourses but also write it in a language whose rhetorical convention is quite different from their own language. (need revise)

Writing a scientific research article is very different from the regular communicating writing (ref). The aim of research articles is presenting a particular information in a structured way, and readers can understand this article more effectively. Therefore, research articles are structured according to a certain rhetorical pattern (ref). Experienced writers already have a rhetorical pattern in mind and use them in their writing; however, novice writers lack that knowledge. To cope with this challenge, we design a writing tool that helps novice writers to produce a better research article by incorporating a rhetorical moves structure. By making the knowledge of writing explicit to novice writer through crowds genre analysis and a structure interface, writers can learn on the spot and produce a better result. 

\section{Day 6: Learnersourcing for writing}
In an education setting, learners are intrinsically motivated to contribute their efforts as they can learn the skill in the process. In this work, we build a learnersourcing platform which allows learners to annotate sentence structure of a document and learn the writing skill from those writing examples. They also can submit their writing to the platform for obtaining feedback from other learners. 

\section{Day 5: Extracting Feedback with Global Concerns}
External Feedback helps a writer improve his/her writing. Obtaining feedback from experts is useful but limited due to the availability of experts. Prior studies have explored a scalable way by leveraging online crowd workers to provide diverse comments. However, novice workers tend to make suggestions for surface errors rather than global issues such as coherence or organization. To improve the organization of a document, we require more diverse feedbacks with global concerns. In this work, we design some reading comprehension questions to guide novice readers to understand the transition between ideas and provide comments for improving the development.

\section{Day 4: Crowdsourced Summary}
A well-written summary helps readers understand the main ideas in a document with less time and effort. However, generating a high-quality summary is difficult for human and machine. Humans need to spend some time to read through the content, digest all of the information in a document, and generate a summary by synthesizing selected ideas. On the other hand, machines can identify key sentences by a statistical algorithm, but still cannot produce a coherent summary without human intervention. In this project, we attempt to combine the power of humans and machines to produce a high-quality summary from an academic paper. Machines can identify key sentences, and human can generate a summary by properly organizing those key sentences.

\section{Day 3: Writing Process}
Writing is a complex process which can be broken into a series of steps. The steps are pre-writing, drafting, revising, editing, and publishing. Pre-writing is the planning phase of the writing process including brainstorming, collecting ideas, and creating an outline to organize ideas. Drafting is to convert ideas into paragraphs. Revising is the phase of modifying ideas and reorganizing structure by adding, deleting, or rearranging content. The goal of this stage focuses on fixing global issues in writing. Editing is the error-correction phase of the process. At this point, writers proofread and correct surface errors such as grammar, typos, and other mechanics. Instead of fixing global issues, this phase focuses on local issues. After finishing previous steps, writers can publish their work at the end of the process.

\section{Day 2: Writing Structure Extraction}
The quality of paper structure affects reading comprehension. To help a writer improve she/his writing, we need to assess the structure of a paper. However, evaluating the structure of a document is challenging for a novice reader. It allows a reader to understand, analyze, and judge ideas delivered in a paper. To aid structure assessment, we provide an approach which identifies the main ideas and extracts a causal relation between two ideas.

\section{Day 1: Crowdsourced Writing}
Recent studies have decomposed writing process into a series of micro-tasks in crowdsourcing domain. Soylent \cite{Bernstein:UIST10} splits writing projects into multiple stages and allows crowd workers to shorten, proofread, and edit a document. MicroWriter [cited] enable collaborative writing by breaking the task of writing into three types of micro-tasks--generating ideas, labeling ideas to organize them, and writing paragraphs given a few related ideas.

\bibliographystyle{abbrv} 
\bibliography{writing100}

\end{document} 